\documentclass[10pt]{article}              % Book class in 11 points
\parindent0pt  \parskip10pt             % make block paragraphs
\usepackage[dvips]{color}

\usepackage{tcolorbox}
\tcbuselibrary{breakable}

\usepackage{afterpage}

\usepackage{xcolor}
\usepackage[basque]{babel}
\selectlanguage{basque}
\usepackage[hmargin=2.5cm,vmargin=2.5cm,headheight=15pt]{geometry}
\usepackage{graphicx}
\usepackage{amsmath,amsfonts,amssymb}
%\usepackage{picins}

\begin{document}                        % End of preamble, start of text.




\section*{2018-2019 Ikasturtea\\
\textit{Termodinamika eta Fisika Estatistikoa irakasgaia}\\
1. azterketa partziala, Termodinamika\\
(2019ko urtarrilaren 25a)}

\vspace{2.5cm}


\begin{enumerate}



\item 
	\begin{enumerate}
	\item Frogatu bero-iturri batekin ukipen termikoan gertatzen den bi bolumenen arteko gas baten zabaltzean, bero gehiago xurgatzen duela gasak zabaltzea itzulgarria bada itzulezina bada baino.
	\item[] 	
	\item Frogatu $V$ eta $T$ konstantepeko prozesu itzulezinaren ondorioz edozein sistemaren energia askeak ($F$) behera egiten duela.
	\item[] 
	\item Uraren, ur likidoaren, $\alpha$ zabalkuntza-koefizientea negatiboa da $0^{\circ} \textrm{C}< T < 4^{\circ} \textrm{C}$ tenperatura-tartean.\\ 
	Frogatu ezen urak beroa xurgatzen duela era isotermo itzulgarrian konprimituz gero $3^{\circ}$ Can.
	\item[] 
	\item Likido bati dagokion lurrun-presioaren adierazpena honako hau da: $\ln p = A - \frac{B}{T} + C\ln T$. Adierazpen horretan $(A, B, C)$ konstanteak dira. Lortu $\Delta H_{l\to g}$.
	\item[] 
	\item Sistema bati dagokion askatasun-gradu batekin lotutako aldagai estentsibo eta intebtsiboa $X$ eta $Y$ dira, hurrenez hurren. Askatasun-gradu horrekin lotutako egoera-ekuazioa honako hau da: $ X = \frac{c}{T} Y $. Adierazpen horretan $c$ da konstante ezaguna. Lortu askatasun-graduarekin lotutako bero-ahalmenen arteko lotura (erlazioa),
	\end{enumerate}
	
\vspace{3cm}

\item 
\begin{enumerate}
\item Lortu gas ideal monoatomiko baten $F\equiv U\left[T\right]=F\left(T,V,N\right)$ potentzial termodinamikoa. Honako hau:  

$$F = N R T \left\{ \frac {F_{0}}{N_{0}RT_{0}} - \ln \left[ \left( \frac {T}{T_{0}} \right) ^ {3/2} \left( \frac {V}{V_{0}} \right) \left(\frac{N}{N_{0}} \right)^{-1} \right] \right\}$$

\item[]

\item Gas ideal monoatomikoren horren 2 mol $(p_{i}, V_{i})$ hasierako oreka-egoeratik $(p_{f}=B^{2}p_{i}, V_{f}=\frac{V_{i}}{B})$ ($B$ konstantea da) amaierako oreka-egoerara eraman ditugu. Foko termikoa ($T_{C}$ tenperaturako bero-iturria) eta lan-fokoa erabilgarriak dira. Lortu lan-fokoari eman diezaiokegun lan maximoa. $B$, $p_{i}$ eta $T_{C}$ parametroen balioak finkaturik daudela, $V_{i}$ bolumenaren zein baliok egingo du lana positibo?
\end{enumerate}

\cleardoublepage

\item Gomazko banda baten egoera-ekuazioa honako hau da:

$$ \tau = a\, T \left[ \left(\frac{L}{L_{0}}\right) - \left(\frac{L_{0}}{L}\right)^{2} \right]$$
Adierazpen horretan, $\tau$ da tentsioa, eta $L$, luzera. $C_{L}$, $a$ eta $L_{0}$ konstante ezagunak dira.
\begin{enumerate}
\item Froga ezazu tenperaturaren funtzioa baino ez dela barne-energia.
\item[]
\item Banda luzatzen da, era isotermo itzulgarrian, $L = L_{0}$-tik $L = 2L_{0}$-ra.\\
Prozesuan tenperatura 300 K da.\\
Lortu bandaren gainean egindako lana $(W)$ eta trukatu behar izan den beroa $(Q)$.
\item[]
\item Banda hori isoentropikoki luzatu izan balitz, zer tenperatura litzateke bukaerakoa?
\end{enumerate}

\vspace{5cm}

\item Aztertu beharreko sistema honako hau da: 1 atm-ean dagoen gordailuan sartu dugun substantzia baten lurruna. Gordailua 400 K-ean dagoen bero-iturriarekin ukipenean jarri dugu eta, tenperatura konstate mantenduz, 10 atm-raino konprimitu da.\\
    
    Ezaguna da substantzia hori 300 K-ean eta 1 atm-eko presioan lurrunduko dela.\\
    
    Lurruntze-prozesuari dagokion entropia-aldaketa da malda negatiboko lerro zuzena, hain zuzen, honako hau: $\Delta s=-0.0676$ (cal/K$^2$ mol) $\times T + 37.856$ (cal/mol K).\\
    
    Substantziaren likidoaren bolumen espezifikoak ondoko egoera-ekuazioari segitu dio: $v=v_{0}(1+aT)$; $a=10^{-6}$ K$^{-1}$.
    
\begin{enumerate} 
\item Irudikatu prozesua $p/T$ diagraman, ezagunak diren puntu guztiak kokatuz.
\item[]
\item Kalkulatu sistemaren entropia-aldaketa.
\item[]
\item Kalkulatu fase-trantsizioan gertatu den barne-energiaren aldaketa.
\item[]
\item Puntu hirukoitzaren tenperatura 200 K bada, nola kalkulatuko zenuke puntu hirukoitzaren presioa? Azaldu.
\end{enumerate}

\end{enumerate}
%



%}
\end{document}
