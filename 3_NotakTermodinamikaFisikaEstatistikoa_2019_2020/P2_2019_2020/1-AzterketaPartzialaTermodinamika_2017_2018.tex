\documentclass[10pt]{article}              % Book class in 11 points
\parindent0pt  \parskip10pt             % make block paragraphs
\usepackage[dvips]{color}

\usepackage{tcolorbox}
\tcbuselibrary{breakable}

\usepackage{afterpage}

\usepackage{xcolor}
\usepackage[basque]{babel}
\selectlanguage{basque}
\usepackage[hmargin=2.5cm,vmargin=2.5cm,headheight=15pt]{geometry}
\usepackage{graphicx}
\usepackage{amsmath,amsfonts,amssymb}
%\usepackage{picins}

\begin{document}                        % End of preamble, start of text.




\section*{2018-2019 Ikasturtea\\
\textit{Termodinamika eta Fisika Estatistikoa irakasgaia}\\
2. azterketa partziala, Estatistika\\
(2019ko maiatzaren 24a)}

\vspace{0.25cm}


\begin{enumerate}


\item {\bf{3 dimentsioko solidoetako akatsen azterketa}} 

\item[] Ikusi dugunez, solidoak ez dira perfektuak; aztertu dugu baita ere 2 dimentsioko zenbait akatsdun solido. Oraingo honetan, 3 dimentsioko akatsdun solidoak aztertuko dituzu. 

Aztertuko duzun solidoa deskribatzeko hiru dimentsioko sarea erabiliko duzu. Sareak dauzka $N_{x}$, $N_{y}$ eta $N_{z}$ sare-puntu $OX$, $OY$ eta $OZ$ ardatz cartesiarretan, hurrenez hurren; eta onartuko duzu atomo bana dutela horietakoek, solido perfektua eratzekotan, behintzat. Hiru akts mota aztertuko dituzu, honako hauek:
 
\begin{enumerate}
	\item[1] Ohiko sare-puntuetan egon beharrean, zenbait atomo, $n$, ohiko sare-puntuen artekoetan kokatzen dira. Horrelako akatsekin lotutako energia $\epsilon$ da. Aldiz, ohiko sare-puntuetan dauden atomoek ez diote solidoaren energiari ekarpenik egiten.  	
	\item[2] Ohiko sare-puntuetako atomo batzuek ($n$) gainazalera ihes egiten dute, hutsuneak utziz eta solidoaren azalera, handituz: nolabait esatearren, solidoaren sare-puntuen kopurua handitzen dutela. Akats hori sorrarazteko energia da $\epsilon$. Ohiko sare-puntuetan dauden atomoek ez diote solidoaren energiari ekarpenik egiten.
	\item[3] Ohiko zenbait sare-puntutan, $n$, ez dago atomorik; hots, solidoak hutsuneak dauzka. Hutsunea sorrarazteko beharrezkoa den energia da $\epsilon$. Ohiko sare-puntuetan dauden atomoek ez diote solidoaren energiari ekarpenik egiten.
\end{enumerate}

Lortu, hiru kasuetan, zenbatekoa den tenperaturako batezbesteko akatsen kopurua: $\langle n \rangle = n(T)$.

\vspace{0.5cm}



\item {\bf{2 dimentsioko eta geruza bakarreko solidoetako molekula polarizagarrien azterketa}} 

\item[] Kasu honetan, bi dimentsioko eta atomo geruza bakarreko solidoa aztertuko duzu, eremu elektriko batean pean. Solido hori $V$ bolumeneko tangan sartu da eta tanga bera bero-iturri batekin ukipenean dago eta, gainera, tanga horretan $A$ eta $B$ partikulez osatutako gas nahastura idela dago (gas ideala bera), $T$ tenperaturan. 

Solidoa deskribatzeko, ohikoa denez, sarea erabiliko duzu. Sare horrek $N$ sare-puntu dauzka eta horietako $n$ sare-puntuk $C$ atomo bana dute. $C$ atomodun $n$ sare-puntuek gas idealeko $A$  motako partikulak, bana aldi berean, xurga ditzateke, polarizagarria den $CA$ molekula eratuz, zeinaren energia den $\epsilon_{CA}$, $A$ partikula xurgatzekotan. Hots, $\vec{p}$ momentu dipolarra dago lotutako $CA$ molekularekin, zeinaren orientazioak sei diren, honako hauek: $\pm p_{x}$, $\pm p_{y}$ eta $\pm p_{z}$. 

Aldiz, $B$ partikulak xurgatzean, ez da polarizagarria den molekularik eratzen. Dena dela, kasu honetan, eta $B$ molekularen tamaina dela eta, bat edo bi $B$ motako atomoak xurga ditzateke $C$ atomoek: bat xurgatzen denean, 3 egoera dira posible eta bi xurgatzen direnean, 2 egoera dira posible. Eratzen den molekularen energia da $\epsilon_{CB}$.

Hori horrela, martxan ezarri da kanpo eremu elektriko bat: ${\large{\vec{\varepsilon}}}={\large{\varepsilon}}\cdot \vec{u}_{x} $. 

\begin{enumerate}
\item[1] Lortu partizio-funtzio gran kanonikoa.
\item[2] Lortu $\langle n \rangle$ estaltzearen adierazpena.
\item[3] Lortu sistemaren polarizazioa.
\item[4] Azaldu, eta lortu, nola kalkulatuko zenukeen partizio-funtzio kanonikoa.


\end{enumerate}

\vspace{0.5cm}

\item Aurreko solidoko (bi dimentsiokoa, $A$ azalerakoa, geruza bakarrekoa eta atomo banako $N$ sare-puntukoa bera) atomoek elektroi bana \textit{askatzen dute} bi dimentsioko $\frac{1}{2}$ spineko fermioiz osatutako gasa ideala eratzeko. Nolakoa da gas horren $\mu$ potentzial kimikoa $T$ tenperaturaren funtzioan?


\vspace{0.5cm}

\item Fermiren gas bat, $\frac{1}{2}$ spineko partikulez osatutako bera, $T=0$ K egoera termikoan dago, $V_{0}$ bulumeneko tangan, adiabatikoki isolatuta. Tangaren hormetako bat bat-batean kendutakoan, gasa era askean zabaltzen da (zabaltze askea da, beraz), $\Delta V_{0}$ ($\frac{\Delta V_{0}}{V_{0}}\lll 1$) bolumenera hedatuz; guztiz hedatutakoan, egoera termiko berri batean dago. Balioztatu egoera termiko berriaren tenperatura.
    
\vspace{0.5cm}

\item

\begin{itemize}
\item[] \textit{\textbf{Partikula-banaketak, kualitatiboki...}} 

\item[] Bete ezazu ondoko taula behar diren adierazpide grafikoak (kualitatibo) irudikatuz.\\
Kontuan izan, kasu bakoitzean, zer egoeren dentsitate erabili behar den, translazioak dira gasen askatasun-graduak, eta zer partikula mota den, fermioia edo bosoia.

\item[]
\begin{figure}[h]
\begin{center}
\includegraphics[width=4in,angle=0]{kuestioak1irudia.pdf}
\end{center}
\end{figure}

\vspace{0.5cm}
\item[] \textit{\textbf{Partikula-banaketak, kuantitatiboki...kasu errazetan.}} 
\item[] Lor itzazu posibleak diren mikroegoera denak (adieraz itzazu grafikoki, era eskematikoan), (i) partikula bereizgarriak, (ii) fermioiak, (iii) bosoiak eta (iv) \textit{partikula klasiko}ak dituzula onartuz, honako bi kasuetan:

\begin{enumerate}
\item bi partikula eta bi egoera posible.
\item hiru partikula eta hiru egoera posible.
\item[] Aldera itzazu lortutako emaitzak eta azaldu.
\end{enumerate}
\end{itemize}


\end{enumerate}
%



%}
\end{document}
