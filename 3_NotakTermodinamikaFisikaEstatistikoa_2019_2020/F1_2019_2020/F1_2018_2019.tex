\documentclass[10pt]{article}              % Book class in 11 points
\parindent0pt  \parskip10pt             % make block paragraphs
\usepackage[dvips]{color}

\usepackage{tcolorbox}
\tcbuselibrary{breakable}

\usepackage{afterpage}

\usepackage{xcolor}
\usepackage[basque]{babel}
\selectlanguage{basque}
\usepackage[hmargin=2.5cm,vmargin=2.5cm,headheight=15pt]{geometry}
\usepackage{graphicx}
\usepackage{amsmath,amsfonts,amssymb}
\usepackage{mathrsfs}  

%\usepackage{picins}

\begin{document}                        % End of preamble, start of text.




\section*{2018-2019 Ikasturtea, 
\textit{Termodinamika eta Fisika Estatistikoa}\\
Ohiko deialdia, 
2019ko ekainaren 7a}

%\vspace{0.25cm}
%%%%%%%%%%%%%%%%%%%%%%%%%%%%%%%%%%%%%%%% termodinamika
\section*{Termodinamika}

\begin{enumerate}
% Termo, 1. eta 2. ariketak


\begin{tcolorbox}[breakable, drop shadow,enhanced,notitle,boxrule=0pt,colback=blue!15,colframe=black!100]

\item[]

Benzenoaren (C$_{6}$H$_{6}$; masa molekularra 78 g) lurruntze-tenperatura 80$^{\circ}$C da, presioa 1 atm denean eta, baldintza horietan, benzenoari dagokion lurruntze-beroa 86 cal/g da.   

Bestalde, benzeno likidoaren eta gasaren bolumen espezifikoak, aipatutako baldintzatan, $v_{l}=1.2$ cm$^{3}$/g eta $v_{g}=356$ cm$^{3}$/g dira, hurrenez hurren.  

Esku artean benzenoaren 1 g duzu, oso-osorik gas egoeran dagoena eta $C_{V}=3.3$ cal/K-eko bero-ahalmeneko eta 500 K-ean dagoen sistemarekin batera.  

Benzenoa kutxa itxian sartu da. Benzenoa eta sistema makina itzulgarriaren bidez termikoki konektatu dira; makina martxan jarri da eta horrela mantendu da benzenoa erabat likidotu den arte.
\\
Lortu honako hauek:

\vspace{0.25cm}
\textbf{1. Ariketa}

\begin{enumerate}
\item[] Zer nolako makina itzulgarria da? 
\item Onartu sistemaren fasea ez dela aldatuko;  \\
  zenbatekoa da bere amaierako tenperatura, $\Delta T^{sis}$?  
\item Benzenoaren eta sistemaren entropia-aldakuntzak: $\Delta S^{\text{ben}}$, $\Delta S^{\text{sis}}$.\\
\end{enumerate}

%\vspace{0.25cm}
\textbf{2. Ariketa}

\begin{enumerate}
\item Makinak egin duen lana: $W$.
\item Prozesua bukatutakoan, benzenoaren: entalpia-aldaketa ($\Delta H^{\text{ben}}$), barne-energiaren aldaketa ($\Delta U^{\text{ben}}$), \textit{Helmholtz-en} funtzioaren aldaketa ($\Delta F^{\text{ben}}$), \textit{Gibbs-en} funtzioaren aldaketa ($\Delta G^{\text{ben}}$).  

\end{enumerate}


\end{tcolorbox}



% Termo, 3. eta 4. ariketak
\item[]

Aurreko ariketako prozesua bukatutakoan, benzenoa dagoen kutxa 1 atm-ean eta 60$^{\circ}$C-an dauden presio-iturriarekin eta bero-iturriarekin kontaktuan dagoen eta $V=35$ $l$-koa den hustutako gordailuan jarri dugu kutxa itxi batean. Kutxa hautsi da. Benzeno guztia lurrundu baino lehen neurtu den presioa 0.008 atm da. Benzeno likidoari dagokion zabalkuntza-koefizientea honako hau da: $\alpha = 10^{-3}$ K$^{-1}$. Benzenoaren lurruna gas idealtzat hartu daiteke.


Lortu honako hauek:

\vspace{0.25cm}
\textbf{3. Ariketa}

\begin{enumerate}
\item Zenbat benzeno lurrundu da aipatutako presioa neurtu denean?  
\item Zenbatekoa da bezenoari (elkarren arteko orekan dauden bi faseez, likidoa eta gasa, osatutako nahasturari) dagokion bolumen espezifikoa?\\
\end{enumerate}

%\vspace{0.25cm}
\textbf{4. Ariketa}

Benzeno guztia lurrundutakoan gordailuan neurtu dugun presioa 0.01 atm da.  

\begin{enumerate}
\item Irudikatu prozesua (osoa, aurreko ariketakoa ere bai) $p/T$ diagraman.
\item Lortu honako hauek: entalpia-aldaketa ($\Delta H^{\text{ben}}$), barne-energiaren aldaketa ($\Delta U^{\text{ben}}$), \textit{Helmholtz-en} funtzioaren aldaketa ($\Delta F^{\text{ben}}$), \textit{Gibbs-en} funtzioaren aldaketa ($\Delta G^{\text{ben}}$).\end{enumerate}



%%%%%%%%%%%%%%%%%%%%%%%%%%%%%% Estatistika
\section*{Estatistika}


%Porpietate komuna.
%  Ariketak denetarik fotokopiak, internetetik ateratakoak
%  Physics 5524, Statistical Mechanics, Problem Set 11, ariketa 11.1 eta 11.2 aldiberean jarrita
%%%%%%%%%%%%%%%%%%%%%%%%%%%%%%%%%%%%%


\item[] \textit{\textbf{Lortuko duzun egoera-ekuazioak ez dauka partikularen izaerarekiko mendekotasunik:\\dela fermioia, dela bosoia, dela partikula klasikoa.}} \\

Har ezazu aintzakotzat fotoiz osatutako gasa, $V$ bolumeneko barrunbean dagoena, $T$ tenperaturan.\\

\textbf{1. Ariketa}

\begin{enumerate} 
\item Idatz ezazu fotoizko gas horren partizio-funtzio gran-kanonikoa: $\mathscr{Z}(T,V;\mu=0)=\mathscr{Z}(T,V)$.
\item Lortu gasari dagokion gran-potentzialaren adierazpen bat.\\
      Adieraz ezazu zure emaitza $T$, $V$ eta $\omega$ maiztasunarekiko integral baten funtzioan.\\
      (Ez duzu integrala egin behar.)
\item Lortu gasari dagokion $E$ energiaren adierazpena.
\item[] Berebat, berori adieraz ezazu $T$, $V$ eta $\omega$ maiztasunarekiko integral baten funtzioan.\\
      (Ez duzu integrala egin behar.)
\item Aurreko bi emaitzak erabilita, lortu $p\,V$ biderkadurari dagokion adierazpena.
\item[] (Kasu honetan integrala kalkulatu behar duzu.)
\end{enumerate} 

\begin{tcolorbox}[breakable,notitle,boxrule=0pt,colback=blue!15,colframe=black!100]
\item[]

\noindent Oraingo honetan, aldiz, aztertuko duzun sistema honako hau da: ultra-erlatibistak ($m\,c^{2}\ll c\,p_{F}$) diren elkarrekintzarik gabeko (independente) elektroiz osatutako gasa. Aipatutako limitean, honako hau da elektroiaren energiaren eta bere momentuaren arteko erlazioa: $\mathscr{\epsilon}(\vec{p})=c\,|\,\vec{p}\,|$.\\
Horrelako $N$ elektroi daude $V$ bolumenean. Spina kontuan hartu behar da egoerak zenbatzeko.

\textbf{2. Ariketa}

\begin{enumerate} 
\item Lortu gasaren $\mu$ potentzial kimikoa eta \textit{Fermi}ren momentua, $p_{F}$, $T=0$ K denenean eta $N$ eta $V$ aldagaien funtzioan.

\item Lortu sistemaren $E$ energia osoa, $T=0$ K denenean eta $N$ eta $V$ aldagaien funtzioan.

\item Lortu, $T=0$ K denean, $p\,V$ biderkadurari dagokion adierazpena.\\
      (Kasu honetan integrala kalkulatu behar duzu.)
\end{enumerate} 

\textbf{3. Ariketa}

\begin{enumerate} 
\item Idatz ezazu elektroizko gas horren partizio-funtzio gran-kanonikoa, tenperatura finituan.\\
      Erabil ezazu emaitza hori gasaren potentzial gran-kanonikoaren adierazpen bat lortzeko.\\
      Adieraz ezazu $T$, $V$, $\mu$ eta energiarekiko integral baten funtzioan.\\
      (Ez duzu integrala egin behar.)
\item Lortu gasari dagokion $E$ energiaren adierazpena.\\
      Berebat, berori adieraz ezazu $T$, $V$, $\mu$ eta energiarekiko integral baten funtzioan.\\
      (Ez duzu integrala egin behar.)

\item Aurreko bi emaitzak erabilita, lortu $p\,V$ biderkadurari dagokion adierazpena.

\end{enumerate} 
\end{tcolorbox}

\textbf{4. Ariketa}
\begin{enumerate} 

\item[] Zein da $\mathscr{\epsilon}(\vec{p})=c\,|\,\vec{p}\,|^{s}$ dispertsio-erlazioko $D$ dimentsioko espazioan bizi den gas baten $p\,V$ biderakaduraren adierazpena?
\end{enumerate} 



\end{enumerate}


\end{enumerate}
%



%}
\end{document}
